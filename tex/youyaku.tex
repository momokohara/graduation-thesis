\documentclass[../main]{subfiles}
\setcounter{secnumdepth}{1}
\begin{document}
 \chapter*{要約}
 \begin{center}
    {\LARGE トポロジカルマップを用いた\\シナリオによるナビゲーション\\
    \vspace{0.5cm}
    −全天球カメラ画像に基づく通路分類手法の提案−}\\
 \end{center}\vspace{0.8cm}


    人の移動する能力をロボットの自律移動に応用する手法として,先行研究によりトポロジカルマップとシナリオを用いた
    ナビゲーション手法が提案された.人は道案内により移動する際,三叉路や突き当たりなどの通路の特徴を重視していることから,この手法では
    ナビゲーションを行いながら通路の特徴を分類し,ナビゲーションに利用する.
    しかし,この手法の問題点としてナビゲーションで必要となる通路の分類に失敗することにより,目的地までたどり着くことができずにナビゲーションに
    失敗するということがあった.
    通路の分類は,Chenらが提案するLiDARを用いた通路検出手法(Toe-Finding Algolithm)\cite{toe-finding_paper}を参考にしており,
    LiDARの周囲に壁などの遮蔽物がなく,開けている方向があればその方向に通路があると検出する.そのため,開いているドアや隙間などを通路と誤検出してしまった.
    そこで,本研究では通路を分類する手法として,全天球カメラ画像を用いた手法を検討する.
    具体的には,全天球カメラで水平360度の画像を取得し,取得した画像からYOLOの学習器を用いて通路やドアなどの物体を検出する.
    また,検出した通路とその方向からどの通路の特徴に相当するかを分類する手法である.本研究では,先行研究と同様の9種類の通路の特徴を分類する.
    本手法で全天球カメラを用いた理由は,全天球カメラはカメラの周囲360度の画像データを一度の撮影により取得することができるため,
    複数台のカメラを使わずに通路分類に必要となるデータを取得することができるからである.
    初めに,通路の画像を集めたデータセットを用いてトレーニングを行い,学習モデルを作成する.次に,YOLOにより検出した画像中の通路の位置座標に基づき,通路の特徴を
    分類するシステムを作成した.最後に,提案した手法により実環境において通路の分類ができるかどうかを検証する.\\\\

    キーワード:トポロジカルマップ,機械学習,道案内
\end{document}