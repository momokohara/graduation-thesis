\documentclass[../main]{subfiles}
\setcounter{secnumdepth}{1}
\begin{document}
 \chapter*{概要}
    {\LARGE トポロジカルマップを用いたシナリオによるナビゲーション\\
    ー全天球カメラ画像に基づく通路認識手法の提案ー}\\
    人の道案内のようにロボットを目的地まで移動させる手法として,先行研究によりトポロジカルマップとシナリオを用いた
    ナビゲーション手法が提案された.この手法の問題点として,ナビゲーションで必要となる通路の特徴抽出に失敗することにより,
    ナビゲーションにも失敗するということがあった.通路の特徴抽出にはLiDARを用いており,センサが開いているドアなどに反応してしまうことで誤認識が発生した.
    そこで,本研究では通路の特徴情報を抽出する手法として,全天球カメラ画像を用いた手法を検討する.
    具体的には,YOLOを用いて全天球カメラ画像から通路を検出し,画像中の通路の位置座標に基づくことにより通路の特徴を認識する.
    初めに,通路のデータセットを用いて学習モデルを作成し,この手法により通路の特徴を検出できるか検証を行う.
    その後,先行研究のトポロジカルマップとシナリオを用いたナビゲーションに提案手法を適用し,ナビゲーションの成功回数を先行研究の結果と比較することにより,提案した手法の
    有効性の検証を行う.
    
    キーワード:
    トポロジカルマップ,機械学習,道案内
\end{document}