\documentclass[../main]{subfiles}
\begin{document}
\setcounter{secnumdepth}{3}
    \chapter{要素技術}
        \section{トポロジカルマップ}
        私たちの身の回りには様々な種類の地図があり,活用されている.
        例えば,Fig~に表すメトリックマップと呼ばれる地図は,普段人が目的地まで移動する際に用いられる.
        しかし,本研究で用いているトポロジカルマップはFig~のような形をしている.メトリックマップがやや複雑な形をしているのに対し,
        トポロジカルマップはより簡潔に,環境を抽象的に表現することができる.

        
        トポロジカルマップは,大きく分けてノードとエッジの2つの要素により構成されている.
        Fig~では,赤い丸の図形で表現されているのがノードである.ノードには,地図の作成者が好きな情報を入れることができる.
        もう1つの要素であるエッジは,それぞれのノード同士を接続するのに用いられる.
        ノード同士に関係性がある場合,ノードとノードはエッジにより接続される.
        \section{Neural Network}
        \section{Convolutional Neural Network(CNN)}
        \section{YOLO}
\end{document}
