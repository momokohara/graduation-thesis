\documentclass[../main]{subfiles}
\begin{document}
\setcounter{secnumdepth}{1}
    \chapter{トポロジカルを用いたシナリオによる\\ナビゲーション}
    \vspace{1cm}
    先行研究のトポロジカルマップとシナリオを用いたナビゲーションでは,\fref{figure::topologicalmap_simada}に示すようなトポロジカルマップと
    道案内を文章で表現したシナリオを用いた.
    人の道案内を模倣する手法であるため,人がどのような情報をもとに移動しているのかを調査するための人の道案内に関するアンケートが実施された.
    このアンケートにより,人は道案内により移動する際に向いている方向と三叉路や突き当たりのような通路の特徴を重視しているということがわかった.
    そのため先行研究のトポロジカルマップとシナリオは,それらの情報を保持できるような形式で作成された.
    トポロジカルマップは,

\end{document}