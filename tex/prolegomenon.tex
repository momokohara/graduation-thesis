\documentclass[../main]{subfiles}
\begin{document}
    \setcounter{secnumdepth}{2}
    \chapter{序論}
        \section{背景}
         近年,人の移動する能力をロボットの自律移動に応用する手法が研究されている.
        例えば,島田らは人の道案内に注目し,道案内のアンケートを基にナビゲーションに用いるトポロジカルマップと
        シナリオの形式を提案し,それらを用いた実ロボットによるナビゲーションの有効性を検証した.
        この研究では,通路の認識が正しく行われた場合は,提案したナビゲーション手法により目的地に到達できるが,
        誤認識が起きた場合はロボットが経路から外れ,ナビゲーションに失敗してしまうということが報告されている.
        先行研究では,通路の認識にはLiDARを使用しており,通路の誤認識は,開いているドアや隙間にLiDARが反応したことが原因であると述べられている.
        ここで,通路の認識にカメラ画像を用いることで,誤認識を解消し,ナビゲーション途中に経路から外れるという問題を解決できるのではないかと考えた.
        \section{目的}
         本研究は,全天球カメラ画像に基づく通路認識の手法を提案する.そして,先行研究で提案された,実ロボットを用いたトポロジカルマップと
        シナリオに基づくナビゲーションに適用することにより,その有効性を検証する.また,検証はナビゲーションの成功回数を先行研究の結果と比較することにより行う.
        \section{関連研究}
        \section{本論文の構成}
         本論文ではまず,第1章で研究背景,目的,関連研究について述べた.第2章では,本研究で用いる要素技術について述べる.また,第3章では
\end{document}