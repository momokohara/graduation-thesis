\documentclass[../main]{subfiles}
\begin{document}
\setcounter{secnumdepth}{2}
    \chapter{実験}
    \section{実験の手順}
    本研究では,提案した通路認識手法について2つのフェーズに分けて検証していく.
    まず1つ目のフェーズにおいて,提案した手法による通路認識が可能かどうかを検証する.次に,2つ目のフェーズで,
    提案手法をトポロジカルマップとシナリオによるナビゲーションに適用し,手法の有効性について検証する.
    実験環境は,に示すように千葉工業大学津田沼キャンパス2号館3階の廊下とした.
    \section{実験装置}
    本研究で使用した実験装置をに示す.
    使用したPCのスペックを\tref{table::pc_spec}に示す.
    \begin{table}[H]
        \centering
        \begin{tabular}{l|l} \hline
        \multicolumn{1}{c|}{CPU} & \multicolumn{1}{c}{Core i7-9750H(Intel)} \\ \hline
        \multicolumn{1}{c|}{RAM} & \multicolumn{1}{c}{16GB} \\ \hline
        \multicolumn{1}{c|}{GPU} & \multicolumn{1}{c}{RTX 2070 Max-Q} \\ \hline
        \end{tabular}
        \caption{Specification of PC.}
        \label{table::pc_spec}
    \end{table}

    \section{実験目的}
    本研究により提案した,全天球カメラ画像に基づく通路認識の手法により通路の認識ができるかを検証すること.
    \section{実験方法}

    \section{結果}
    \section{考察}
\end{document}