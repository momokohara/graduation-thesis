\documentclass[../main]{subfiles}
\begin{document}
\setcounter{secnumdepth}{2}
    \chapter{まとめ}
        \section*{謝辞}
        本研究を進めるにあたり,熱心にご指導を頂いた林原靖男教授に深く感謝いたします.また,島田先輩には研究を引き継がせていただき,
        多くの知識や経験をもとに研究のサポートをしていただきました.また,高橋先輩にも多くのサポートをいただきました.
        日常の議論を通じて多くの知識や示唆を頂いたロボット設計制御研究室の皆様に謝意を表します.
        \addcontentsline{toc}{section}{謝辞}
        %\section*{参考論文}
        %\addcontentsline{toc}{section}{参考論文}
        %\begin{thebibliography}{7}
            %\bibitem{yolo_paper_fig}Joseph Redmon, Santosh Divvala, Ross Girshick, Ali Farhadi, "You Only Look Once: Unified, Real-Time Object Detection", 2015年, 1ページ
            %\bibitem{yolo_paper_v1} Joseph Redmon, Santosh Divvala, Ross Girshick, Ali Farhadi, ”You Only Look Once: Unified, Real-Time Object Detection”, arXiv:1506.02640[cs.CV](2015)
            %\bibitem{yolo_paper_v2} Joseph Redmon, Ali Farhadi, "YOLO9000: ", arXiv:1612.08242[cs.CV] (2016)
            %\bibitem{yolo_paper_v3} Joseph Redmon, Ali Farhadi, "YOLOv3: An Incremental Improvement", arXiv:1804.02767[cs.CV] (2018)
            %\bibitem{yolo_paper_v4} Alexey Bochkovskiy, Chien-Yao Wang, Hong-Yuan Mark Liao, "YOLOv4: Optimal Speed and Accuracy of Object Detection", arXiv:2004.10934[cs.CV] (2020)
            %\bibitem{shimada_paper1} 島田滉己,上田隆一,林原靖男,”トポロジカルマップを用いたシナリオによるナビゲーションの提案 ー人が道案内に用いる情報の取得と評価ー”,日本機械学会ロボティクス・メカトロニクス講演会'20予稿集,2P1-K02(2020)
            %\bibitem{shimada_paper2} 島田滉己,上田隆一,林原靖男,”トポロジカルマップを用いたシナリオによるナビゲーションの提案 ーシナリオに基づく実ロボットのナビゲーションー”,1H2-04,SI2020(2020)
        %\end{thebibliography}
        %\section*{付録資料}
        %\addcontentsline{toc}{section}{付録資料}
\end{document}